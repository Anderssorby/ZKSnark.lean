\documentclass{article}
\usepackage[utf8]{inputenc}
\usepackage{amsmath} % Required for some math elements
\usepackage{amsfonts}
\usepackage{amsthm} % for proof pkg
\usepackage{amssymb}
\usepackage{verbatim}
\usepackage{hyperref}
\usepackage{comment}
\usepackage{color}
\usepackage{parskip} % Remove paragraph indentation
\usepackage[margin=1in]{geometry}
\usepackage[inline]{enumitem}
\usepackage{mathtools}
\usepackage{multirow}
\usepackage{listings}

\newtheorem{lemma}{Lemma}
\newtheorem{claim}{Claim}
\newtheorem{theorem}{Theorem}

\theoremstyle{definition}
\newtheorem{definition}{Definition}[section]
\newtheorem{fact}{Fact}[section]
\theoremstyle{remark}
\newtheorem*{remark}{Remark}

\newcommand{\anote}[1]{{\color{magenta} [AM: #1]}}
\newcommand{\Gen}{\mathsf{Gen}}
\newcommand{\Enc}{\mathsf{Enc}}
\newcommand{\Dec}{\mathsf{Dec}}
\newcommand{\pk}{\mathit{pk}}
\newcommand{\sk}{\mathit{sk}}
\newcommand{\bbZ}{\mathbb{Z}}
\newcommand{\bit}{\{0,1\}}
\newcommand{\la}{\leftarrow}
\newcommand{\ninN}{{n \in \mathbf{N}}}
\newcommand{\cF}{\mathcal{F}}
\newcommand{\cG}{\mathcal{G}}
\newcommand{\RF}{\mathsf{RF}}
\newcommand{\Half}{\frac{1}{2}}
\newcommand{\F}{\mathbb{F}}
\newcommand{\Adv}{\mathcal{A}}
\newcommand{\Ext}{\mathcal{E}}

\newcommand{\ignore}[1]{{}}

\newcommand{\samples}{\overset{\$}{\leftarrow}}
\newcommand{\hash}{\ensuremath{\mathcal{H}}}
\newcommand{\doubleplus}{+\kern-1.3ex+\kern0.8ex}
\newcommand{\mdoubleplus}{\ensuremath{\mathbin{+\mkern-10mu+}}}

\title{Formalising Groth16 in Lean 4}
\author{Daniel Rogozin, for Yatima Inc}
\date{\today\footnote{This document may be updated frequently.}}

\begin{document}

\maketitle

In this document, we describe the Groth16 soundness formalisation in Lean 4.
The text contains the protocol description as well as some comments to its implementation.

\section{Preliminary definitions}

We have a fixed finite field $F$, and $F[X]$ stands for the ring of polynomials over $F$ as usual. The corresponding listing:

\begin{lstlisting}
variable {F : Type u} [field : Field F]
\end{lstlisting}

In Groth16, we have random values $\alpha, \beta, \gamma, \delta \in F$ that we introduce separately as an inductive data type:
\begin{lstlisting}
inductive Vars : Type
  | alpha : Vars
  | beta : Vars
  | gamma : Vars
  | delta : Vars
\end{lstlisting}

We also introduce the following parameters:

\begin{itemize}
\item $n_{stmt} \in \mathbb{N}$ --- the statement size;
\item $n_{wit} \in \mathbb{N}$ --- the witness size;
\item $n_{var} \in \mathbb{N}$ --- the number of variables.
\end{itemize}

In Lean 4, we introduce those parameters as variables in the following way:

\begin{lstlisting}
variable {n_stmt n_wit n_var : Nat}
\end{lstlisting}

We also define several finite collections of polynomials:

\begin{itemize}
\item $u_{stmt} = \{ f_{i} \in F[X] \: | \: i < n_{stmt} \}$
\item $u_{wit} = \{ f_{i} \in F[X] \: | \: i < n_{wit} \}$
\item $v_{stmt} = \{ f_{i} \in F[X] \: | \: i < n_{stmt} \}$
\item $v_{wit} = \{ f_{i} \in F[X] \: | \: i < n_{wit} \}$
\item $w_{stmt} = \{ f_{i} \in F[X] \: | \: i < n_{stmt} \}$
\item $w_{wit} = \{ f_{i} \in F[X] \: | \: i < n_{wit} \}$
\end{itemize}

We introduce those collections in Lean 4 as variables as well:

\begin{lstlisting}
variable {u_stmt : Fin n_stmt -> F[X]}
variable {u_wit : Fin n_wit -> F[X]}
variable {v_stmt : Fin n_stmt -> F[X]}
variable {v_wit : Fin n_wit -> F[X]}
variable {w_stmt : Fin n_stmt -> F[X]}
variable {w_wit : Fin n_wit -> F[X]}
\end{lstlisting}

Let $(r_i)_{i < n_{wit}}$ be a collection of elements of $F$ (that is, each $r_i \in F$) parametrised by elements of $n_{wit}$. Define a polynomial $t \in F[X]$ as:
\begin{center}
$t = \prod \limits_{i \in n_{wit}} (x - r_i)$.
\end{center}

Crearly, these $r_i$'s are roots of $t$. The definition in Lean 4:
\begin{lstlisting}
variable (r : Fin n_wit -> F)

def t : F[X] := Pi i in finRange n_wit,
  (x : F[X]) - Polynomial.c (r i)
\end{lstlisting}

\section{Properties of $t$}

The polynomial $t$ has the following properties:

\begin{lemma}
$ $

\begin{enumerate}
\item $deg(t) = n_{wit}$;
\item $t$ is monic, that is, its leading coefficient is equal to $0$;
\item If $n_{wit} > 0$, then $deg(t) > 0$.
\end{enumerate}
\end{lemma}

We formalise these statements as follows (but we drop proofs):
\begin{lstlisting}
lemma nat_degree_t : (t r).natDegree = n_wit
lemma monic_t : Polynomial.Monic (t r)
lemma degree_t_pos (hm : 0 < n_wit) : 0 < (t r).degree
\end{lstlisting}

Let $\{ a_{{wit}_i} \: | \: i < n_{wit}\}$ and $\{ a_{{stmt}_i} \: | \: i < n_{stmt} \}$ be collections of elements of $F$. A stamenent witness polynomial pair is a pair of single variable polynomials $(F_{{wit}_{sv}}, F_{{stmt}_{sv}})$
such that $F_{{wit}_{sv}}, F_{{stmt}_{sv}} \in F[X]$ and
\begin{itemize}
\item $F_{{wit}_{sv}} = \sum \limits_{i < n_{wit}} a_{{wit}_i} u_{{wit}_{i}}(x)$
\item $F_{{stmt}_{sv}} = \sum \limits_{i < n_{stmt}} a_{{stmt}_i} u_{{stmt}_{i}}(x)$
\end{itemize}

Their Lean 4 counterparts:
\begin{lstlisting}
def V_wit_sv (a_wit : Fin n_wit -> F) : Polynomial F :=
  \sum i in finRange n_wit, a_wit i \bullet u_wit i

def V_stmt_sv (a_stmt : Fin n_stmt -> F) : Polynomial F :=
  \sum i in finRange n_stmt, a_stmt i \bullet u_stmt i
\end{lstlisting}


Define a polynomial $sat$ as:
\begin{multline}
sat = (V_{{stmt}_{sv}} + V_wit_sv) \cdot \\ \cdot ((\sum \limits_{i < n_{stmt}} a_{{stmt}_i} v_{{stmt}_i}(x)) + (\sum \limits_{i < n_{wit}} a_{{wit}_i} v_{{wit}_i}(x) )) - \\ - ((\sum \limits_{i < n_{stmt}} a_{{stmt}_i} w_{{stmt}_i}(x)) + (\sum \limits_{i < n_{wit}} a_{{wit}_i} w_{{wit}_i}(x) ))
\end{multline}
A pair $(F_{{wit}_{sv}}, F_{{stmt}_{sv}})$ satisfies the square span program, if the remainder of division of $sat$ by $t$ is equal to $0$.

The Lean 4 analogue of the property defined above:
\begin{lstlisting}
def satisfying (a_stmt : Fin n_stmt -> F) (a_wit : Fin n_wit -> F) :=
  (((\sum i in finRange n_stmt, a_stmt i \bullet u_stmt i)
    + \sum i in finRange n_wit, a_wit i \bullet u_wit i)
  *
  ((\sum i in finRange n_stmt, a_stmt i \bullet v_stmt i)
    + \sum i in finRange n_wit, a_wit i \bullet v_wit i)
  -
  ((\sum i in finRange n_stmt, a_stmt i \bullet w_stmt i)
    + \sum i in finRange n_wit, a_wit i \bullet w_wit i) : F[X]) %_m (t r) = 0
\end{lstlisting}

\section{Common reference string elements}

Assume we interpreted $\alpha$, $\beta$, $\gamma$, and $\delta$ somehow with elements of $F$, say $crs_{\alpha}$, $crs_{\beta}$, $crs_{\gamma}$, and $crs_{\delta}$, that is, in Lean 4:

\begin{lstlisting}
def crs_alpha  (f : Vars -> F) : F := f Vars.alpha

def crs_beta (f : Vars -> F) : F := f Vars.beta

def crs_gamma (f : Vars -> F) : F := f Vars.gamma

def crs_delta (f : Vars -> F) : F := f Vars.delta
\end{lstlisting}

For simplicity, we write this interpretation as a function $f :  \{ \alpha,\beta,\gamma, \delta \} \to F$ defined by equations:
\begin{center}
$f(a) = crs_{a}$ for $a \in \{ \alpha,\beta,\gamma, \delta \}$.
\end{center}

In addition to those four elements of $F$ we have a collection of degrees for $a \in F$/
\begin{center}
$\{ a^i \: | \: i < n_{var} \}$
\end{center}
formalised as:
\begin{lstlisting}
def crs_powers_of_x (i : Fin n_var) (a : F) : F := (a)^(i : Nat)
\end{lstlisting}

We also introduce collections $crs_l$, $crs_m$, and $crs_n$ for $a \in F$:
\begin{multline}
crs_l = \frac{((f(\beta) / f(\gamma)) \cdot (u_{{stmt}_i})(a)) + ((f(\alpha) / f(\gamma)) \cdot (v_{{stmt}_i})(a)) + w_{{stmt}_i}(a)}{f(\gamma)} \\ \text{for $i < n_{stmt}$}
\end{multline}

\begin{multline}
crs_l = \frac{((f(\beta) / f(\delta)) \cdot (u_{{wit}_i})(a)) + ((f(\alpha) / f(\delta)) \cdot (v_{{wit}_i})(a)) + w_{{wit}_i}(a)}{f(\delta)} \\ \text{for $i < n_{wit}$}
\end{multline}

\begin{center}
$crs_l = \frac{a^i \cdot t(a)}{f(\delta)}, \text{for $i < n_{var}$}$
\end{center}

Their Lean 4 version:

\begin{lstlisting}
def crs_l (i : Fin n_stmt) (f : Vars -> F) (a : F) : F :=
  ((f Vars.beta / f Vars.gamma) * (u_stmt i).eval (a)
  +
  (f Vars.alpha / f Vars.gamma) * (v_stmt i).eval (a)
  +
  (w_stmt i).eval (a)) / f Vars.gamma.

def crs_m (i : Fin n_wit) (f : Vars -> F) (a : F) : F :=
  ((f Vars.beta / f Vars.delta) * (u_wit i).eval (a)
  +
  (f Vars.alpha / f Vars.delta) * (v_wit i).eval (a)
  +
  (w_wit i).eval (a)) / f Vars.delta

def crs_n (i : Fin (n_var - 1)) (f : Vars -> F) (a : F) : F :=
  ((a)^(i : Nat)) * (t r).eval a / f Vars.delta
\end{lstlisting}

\end{document}
